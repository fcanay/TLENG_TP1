%TODO no estan hechas!

Para concluir, nos parece importante aclarar que ningún scheduler es el scheduler definitivo, sino que cada uno tiene sus fortalezas y debilidades. Va a depender del contexto de uso y del programador que implemente el algoritmo del scheduler utilizar uno u otro.

Por ejemplo, FCFS es muy bueno en términos de minimizar el tiempo entre comenzar y terminar las tareas, o si el context switch es muy caro, pero su performance empeora cuando las tareas realizan muchas llamadas bloqueantes.

Por otro lado, si tenemos un sistema en tiempo real probablemente utilicemos el algoritmo Relative urgency.