\subsection{Decisiones de diseño}
Implementamos el \emph{AFD} como una clase con la siguiente estructura:

\begin{itemize}
	\item estados: Una lista de enteros para representar a los estados del \emph{AFD}
	\item estado\_inicial: Representa al estado inicial del \emph{AFD}
	\item estados\_finales: Una lista de enteros para representar a los estados finales del \emph{AFD}
	\item delta: Es un diccionario que dice, para cada estado, qué tranciciones puede tomar y hacia qué estado va al tomar dichas transiciones. El diccionario tiene estados como claves y para cada estado tiene una lista de pares $<char, estado>$ como significado.
	\item alfabeto: Una lista de caracteres que representa los distintos elementos del lenguaje del \emph{AFD}
\end{itemize}

Para mayor comodidad, decidimos definir el siguiente invariante de clase: los estados se representan con números enteros del 1 a N, donde N es igual a la longitud de la lista de estados de la clase.

Por otro lado, tomamos como transición $\lambda$ a las transiciones que tienen el caracter vacío.

Decidimos que tanto el algoritmo que calcula el complemento de un \emph{AFD} como el algoritmo que lo minimiza no pueden asumir que el \emph{AFD} de entrada es completo. Por lo tanto, al principio de ambos algoritmos, se llama al método completar del grafo que reciben.

\subsection{Algoritmos Utilizados}

Los algoritmos utilizados en el trabajo práctico son la implementación de los que vimos en clase, tanto en la práctica como en la teórica.

Al implementar los algoritmos minimizar y determinizar utilizamos los algoritmos vistos en la clase práctica.

Para transformar de una expresión regular a un \emph{AFD}, nos basamos el algoritmo en la construcción utilizada en la demostración vista en la clase teórica del teorema que refiere a este mismo tema. Lo mismo ocurre con los algoritmos de la intersección y complemento de \emph{AFDs}.

Para implementar el algoritmo que determina equivalencia entre \emph{AFDs} calculamos el complemento de uno de los dos autómatas y lo intersecamos con el segundo. Concluímos que son equivalentes si y sólo dicha intersección resulta ser el autómata que acepta el lenguaje vacío.


\subsection{Problemas encontrados}

Cuando comenzamos a implementar el trabajo práctico nos enfrentamos al problema de cómo representar a los nodos. Esto podía llegar a ser un problema ya que no podemos repetir el nombre de los nodos.

Originalmente, los habíamos nombrado \emph{$q_i$} con $i$ natural. Además manteníamos un invariante de nombrar los nodos $q_1$, $q_2$, $\hdots$, $q_n$, sea n la cantidad de estados del \emph{AFD}. Esto resultó ser demasiada molestia ya que la $q$ formaba parte del nombre.

Por esto, decidimos deshacernos de la $q$ y mantener un invariante similar: nombrar los nodos $1$, $2$, $\hdots$, $n$, siendo n la cantidad de estados del \emph{AFD}. De esta forma, para agregar un nuevo nodo a nuestro AFD simplemente agregamos el entero $\#estados + 1$ a la lista de estados del AFD.


Durante la implementación del método \emph{toDOT} nos encontramos con el problema de cómo imprimir las transiciones $\lambda$, los caracteres espacio, tab y $\backslash$. Decidimos imprimirlos como \emph{lambda}, \emph{espacio},  \emph{$\backslash$t} y $\backslash$, respectivamente.

Cabe aclarar que, por ejemplo para imprimir $\backslash$ vía DOT tenemos que imprimir $\backslash\backslash$. Para esto, tenemos que escapar ambas $\backslash$, por lo que en nuestro archivo .py se lee $\backslash\backslash\backslash\backslash$