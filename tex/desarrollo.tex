\subsection{Decisiones de diseño}
Implementamos el \emph{AFD} como una clase con la siguiente estructura:

\begin{itemize}
	\item estados: Una lista de enteros para representar a los estados del \emph{AFD}
	\item estado\_inicial: Representa al estado inicial del \emph{AFD}
	\item estados\_finales: Una lista de enteros para representar a los estados finales del \emph{AFD}
	\item delta: Es un diccionario que dice para cada estado, que tranciciones puede tomar y hacia que estado va al tomar dichas transiciones. El diccionario tiene estados como claves y para cada estado tiene una lista de pares $<char, estado>$ como significado.
	\item alfabeto: Una lista de caracteres que representa los distintos elementos del lenguaje del \emph{AFD}
\end{itemize}

En nuestro diseño, decidimos mantener el invariante que los estados se representan con números enteros del 1 a longitud(estados). Este invariante no necesariamente se preserva en los métodos de la clase.

En nuestro diseño los estados se representan con numeros enteros, excepto en el metodo \emph{Determinizar} donde se representan de otra manera. %ver seccion x cuando tengamos la seccion que hable de determinizar

Por otro lado, tomamos como transición $\lambda$ a las transiciones que tienen el caracter vacío ('').

\subsection{Algoritmos Utilizados}

Los algoritmos utilizados en el trabajo práctico son la implementación de los que vimos en clase, tanto en la práctica como en la teórica.

\subsection{Problemas encontrados}
